
\documentclass[12pt]{article}

\usepackage[english]{babel}

\usepackage[utf8]{inputenc}

\usepackage{amsmath}

\usepackage{graphicx}

\usepackage{color}

\usepackage{ifpdf}

\usepackage[plainpages=false,colorlinks=true,pdfpagelabels,pdfborderstyle={/S/U/W 1}]{hyperref}
\usepackage{mathtools}
\usepackage{amsthm}
\theoremstyle{definition}
\newtheorem{defi}{Definition}
\newtheorem{satz}[defi]{Satz}
\newtheorem{lemma}[defi]{Lemma}
\newtheorem{cor}[defi]{Korollar}
\newtheorem{bem}[defi]{Bemerkung}
\newtheorem{bsp}[defi]{Beispiel}
\newtheorem{nota}[defi]{Notation}
\usepackage{mathrsfs}
\usepackage{amssymb}
\usepackage{dsfont}
\newcommand{\C}{C_{BS}}
\begin{document}
\tableofcontents 

%%Basic Setup

\section{The Implied Volatility}

It is well known, that the price of an european call option in the Black-Scholes Model is given by:
$$C_{BS}(S;K;T;r,\sigma)=S\Phi(d_+)-Ke^{-rT}\Phi(d_-)$$
$$d_+=\frac{log(S/K)+(r+\sigma^2/2)T}{\sigma\sqrt{T}}$$
$$d_-=d_+-\sigma\sqrt{T}$$
where $\Phi$ denotes the standard normal cumulative distribution function.
Before introducing the concept of implied volatility, we want to present some basic properties of $C_{BS}$in the following Lemma.
\begin{lemma}
\end{lemma}
\emph{Proof:}\\
%%nurnoch abtippen
\qed\\
By assumption our market is arbitrage-free by this fact we can easily see, that we have $S-Ke^{-rT}<C(T,K)<S$.
Furthermore we have $\lim_{\sigma\rightarrow 0}C_{BS}(\sigma)=S-Ke^{-rT}$, $\lim_{\sigma\rightarrow +\infty}C_{BS}(\sigma)=S$ and
$\partial_\sigma C_{BS}>0$.\\
Now we set
$$C(T,K)=\C(S,K,T,r,\sigma(S,K,T,r)).$$
Because of Lemma ?? the equation above has exactly one solution $\sigma(S,K,T,r)$ given by
$$\sigma(S,K,T,r)=\C^{-1}(S,K,T,r,C(T,K)),$$
where $\C^{-1}(S,K,T,r,\cdot)$ is the inverse of $\C$ as a function of $\sigma$.
Such a solution is called Black-Scholes implied volatility. This implies, that implied volatility is not constand.
The following lemma is the reason, why talking about vanilla options is meaningful.
\begin{lemma}
For Put and Call options with the same maturity and the same strike price the implied volatility of the Put Option coincides with the implied volatility of the Call Option.
\end{lemma}
\emph{Proof:}\\
%%nurnoch abtippen
\qed\\
However the observed return distriburions are asymetric(skewed) and fat-tailed (leptokuric) which means, that for different K and T a different amount of options will end in the mony, that Black-Scholes Model would predict.
Because of the one to one correspondence between price and volatility, seen above one could say, that the implied volatility is just a different mathematical expression for the price.
In practice finding the implied volatility is the most common usage of the Black-Scholes formular. This is usefull to detect how market imlied volatility differs from a constant(as assumed in the Black-Scholes Model) and hence to detecthow the observed price distribuion implied by the market differs from log-normal. These observations are used to find models, which fit better to the market.
\section{The Volatility Surface}
The section before gives a good motivation to consider the implied volatility $\sigma(S,K,T,r)$ for different values of K and T which ends up in a surface$\sigma(K,T)$. The volatility surface.
\subsection{Basic properties}
Since we have $\sigma(K,T)$ as a function of $T$ and $K$ we can fix $K$ and analyse $\sigma(K,T)$ as a function in T which expresses the volatility term structure. On the other hand we can fix $T$ and get a function of K which we can use to analyse the differnce from a constant value as explained above. In the followig we will often talk about moneyness. By moneyness the ratio $S/K$ of $log(S/K)$ is ment.
By the no-arbitrage condition of our market we have the following three basic properties of the volatility surface:
\begin{enumerate}
\item the volatility surface is positve, i.e. $\sigma(T,K)>0 \forall K \in [K_{min},K_{max}],T \in [T_{min},T_{max}]$
\item for a fiexed T $\sigma(K,T)$ is a convex function of K and has a lower concavity bound.
\item short term volatilites with positive (resp. negative moneyness)are normally higher (resp. lower) than their long term counterparts and an asymptotic trend is present as $T \rightarrow \infty$, which reflects the convergence of option prices to a stationary distribution.
\end{enumerate}
As stated in 2. $\sigma(K)$ is convex. If the shaoe is symmetic we are talkig about a (volatility) smile ,otherwise(asymmetric) we talk about the (volatility) skew.
%% Beispiel Bilder
%%Weiter oben Beispiele für Volatility Surfaces

%%%%%%%%%%%%%%%%%%%%%%%%%%%%%%%%%%%%%%%%%%%%%%%%%%%%%%%%
%%%Fat tailed stuff and statisical micro and macro interpretation
%%%%%%%%%%%%%%%%%%%%%%%%%%%%%%


\subsection{Dynamics}
Over time the price of the underlying changes and so does the price of the our Call Option $C(T,K)$. Hence our Volatility surface changes over time as well and is therfore stochastic. In the following we introduce two commonly used approaches.\\\\
Stickiness-by-strike:Here we assume that the implied volatility is independent of S, i.e. there is a function F such that $\sigma(S,T,K,r)=F(T,K,r)$.\\
Using the chainrule we get the following:
$$\frac{\partial C}{\partial S}=\frac{\partial\C}{\partial S}(\sigma(T,K))+\frac{\C}{\sigma}(\sigma(T,K))\frac{\partial F}{\partial S}(T,K,r)=\frac{\partial\C}{\partial S}(\sigma(T,K)).$$
So we found an easy formular for the calculation of the Greek $\frac{\partial C}{\partial S}$. We just have to calculate the Black-Scholes delta and evaluate in $\sigma(T,K)$ which is the observed implied volatility. \\\\
Stickiness-by-delta: Here we assume that the implied volatility only depends on the moneyness (and T and r of course), i.e there is a function F such that $\sigma(S,T,K,r)=F(T,S/K,r)$. Computing Delta yields:
$$\frac{\partial C}{\partial S}=\frac{\partial\C}{\partial S}(\sigma(T,K))+\frac{1}{K}\frac{\C}{\sigma}(\sigma(T,K))\frac{\partial F}{\partial S}(T,S/K,r) $$.
As in the stickiness by strike we have that Delta $\frac{\partial C}{\partial S}$ is the Black-Scholes Delta plus a correction term dependend of Vega and inversely proportional to $K$. Since $\sigma(S,T,K,r)$ is negativly skewed it is decreasing in $K$. Hence, F is increasing in S. Thus the correction term is positive.\\

\subsection{The laverage effect and the volatility skew}

\section{Local Volatility}

\subsection{Implied volatility as an "average" of local volatility}


\end{document}