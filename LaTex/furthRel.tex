\section{Further relationships}
\subsection{Local Variance as a Conditional Expectation of Instantaneous Variance}
In the following section we take a look at stochastic volatility models, especially the Heston model, and discover a connection between the local variance and the instantaneous variance. First we recall the definition of the Heston model.The dynamics of the Heston model are given by
$$dS_t = \mu S_tdt + \sqrt{v_t}S_tdW_t^1$$
$$dv_t = \kappa (\theta - v_t)dt
+ \eta\sqrt{v_t}dW_t^2$$
where $\langle W^1,W^2\rangle_t = \rho t$ and $\kappa , \theta , \eta , v_0 >0$. It is advantageous to write the SDE in terms of the forward price $F_{t,T} = S_t \exp{\left(\mu\left( T-t\right)\right)}$:
\begin{align*}
dF_{t,T} &=\exp{\left(\mu\left( T-t\right)\right)}dS_t-\mu\exp{\left(\mu\left( T-t\right)\right)}S_tdt\\
&=\exp{\left(\mu\left( T-t\right)\right)}\left(\mu S_tdt + \sqrt{v_t}S_tdW_t^1\right)-\mu\exp{\left(\mu\left( T-t\right)\right)}S_tdt\\
&= \sqrt{v_t} F_{t,T} dW_t^1
\end{align*}
The undiscounted price of a European option with strike K and maturity T is given by\\
$$ C ( K , T ) = \mathbb{E} \left[ \left( S_T -K\right) ^+\right] $$
and differentiating once with respect to K gives
$$ \p_K C = -\mathbb{E}\left[\theta\left(S_T-K\right)\right]$$
where $\theta (\cdot)$ is the Heaviside function ($0$ if the argument is negative, $1$ everywhere else). Differentiating a second time with respect to K gives
$$ \p_{K^2} C = \mathbb{E}\left[\delta\left(S_T-K\right)\right]$$
where $\delta (\cdot)$ is the Dirac $\delta$ function ($\infty $ at the origin and $0$ everywhere else). Applying It\^{o} 's lemma and $dF_{T,T}=dS_T $ to the terminal payoff gives
$$d\left( S_T-K\right)^+ = \theta\left( S_T-K\right) dS_T+\frac{1}{2}v_TS_T^2\delta\left( S_T-K\right) dT .$$
Now we take expectations of each side and use the fact that $F_{t,T}$ is a martingale:
$$dC=d\mathbb{E} \left[ \left( S_T -K\right) ^+\right] = \frac{1}{2}\mathbb{E}\left[ v_T S_T^2 \delta\left( S_T-K\right)\right] dT.$$
Also, we can write
\begin{align*}
\mathbb{E}\left[ v_T S_T^2 \delta\left( S_T-K\right)\right] &= \mathbb{E}\left[\mathbb{E}\left[ v_T S_T^2 \delta\left( S_T-K\right)\mid S_T = K \right]\right]\\
&=\mathbb{E}\left[v_T\mid S_T = K\right] K^2\mathbb{E}\left[\delta\left(S_T-K\right)\right]\\
&=\mathbb{E}\left[v_T\mid S_T = K\right] K^2\p_{K^2} C.
\end{align*}
Substituting this into the formula above, we get
$$\p_{T} C = \mathbb{E}\left[v_T\mid S_T = K\right]\frac{1}{2}K^2\p_{K^2} C$$
If we express the option price as a function of the forward price, the Dupire equation simplifies to 
$$\p_{T} C = \frac{1}{2} \sigma^2 K^2\p_{K^2} C.$$
Comparing these results gives
$$\sigma^2(K,T,S_0) = \mathbb{E}\left[v_T\mid S_T = K\right].$$
That means the local variance is the risk neutral expectation of the instantaneous variance conditioned on $S_T=K$.